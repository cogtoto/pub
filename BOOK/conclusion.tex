\chapter*{Conclusion}
\addcontentsline{toc}{chapter}{Conclusion}

Nous avons parcouru un long chemin depuis les fondements du $\lambda$-calcul jusqu'aux limites de la calculabilité.
Voici une représentation graphique des dépendances fonctionnelles et conceptuelles entre les différents chapitres de cet ouvrage.

\vspace{1cm}

\begin{center}
\begin{tikzpicture}[
    mindmap,
    concept color=blue!30,
    every node/.style={concept, execute at begin node=\hskip0pt},
    root concept/.append style={
        concept color=blue!50,
        fill=blue!50,
        text width=3cm,
    },
    level 1 concept/.append style={
        level distance=5cm,
        sibling angle=60,
        text width=2.5cm,
    },
    level 2 concept/.append style={
        level distance=3cm,
        sibling angle=45,
    },
    extra concept/.append style={
        concept color=green!30,
        text width=2.5cm,
    }
]

% Main Node: Chapter 1
\node [root concept] (ch1) {Chapitre 1 \\ \textbf{$\lambda$-calcul} \\ \& Réduction}
    child [concept color=purple!30] { node (ch2) {Chapitre 2 \\ \textbf{Types} \\ \& PTS} }
    child [concept color=orange!30] { node (ch3) {Chapitre 3 \\ \textbf{Interprétation} \\ (Scheme/Lisp)} }
    child [concept color=red!30] { node (ch5) {Chapitre 5 \\ \textbf{Résolution} \\ \& Unification} }
    child [concept color=teal!30] { node (ch6) {Chapitre 6 \\ \textbf{Calculabilité} \\ \& Complexité} };

% Secondary Nodes (Level 2 dependencies)
% Ch 4 depends on Ch 3 and Ch 1
\node [extra concept] (ch4) at (4, -4) {Chapitre 4 \\ \textbf{Compilation} \\ (SKI, SECD)};

% Connections
\path (ch3) edge [concept connection] (ch4);
\path (ch1) edge [concept connection] (ch4);

% Additional conceptual links
\draw [concept connection, dashed] (ch5) edge (ch6); % Validating undecidability
\draw [concept connection, dashed] (ch2) edge (ch5); % Logic vs Types (Curry-Howard implicit link)

\end{tikzpicture}
\end{center}

\vspace{1cm}

Chaque chapitre s'appuie sur le formalisme minimaliste du $\lambda$-calcul (Chapitre 1) pour explorer une facette différente de l'informatique théorique :
\begin{itemize}
    \item La \textbf{rigidité} et la sûreté par le typage (Chapitre 2).
    \item L'\textbf{expressivité} et la dynamique par l'interprétation (Chapitre 3).
    \item L'\textbf{efficacité} par la compilation (Chapitre 4).
    \item La \textbf{logique} et la recherche de preuve par la résolution (Chapitre 5).
    \item Les \textbf{limites} de ces formalismes (Chapitre 6).
\end{itemize}

Ainsi se referme notre étude des fondements.
